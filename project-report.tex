\documentclass{article}

%% Page Margins %%
\usepackage{geometry}
\geometry{
    top = 0.75in,
    bottom = 0.75in,
    right = 0.75in,
    left = 0.75in,
}

\usepackage{amsmath}
\usepackage{graphicx}
\usepackage{parskip}

\title{Assembly Project: Columns}

% TODO: Enter your name
\author{Cynthia Rong 1011129832 and Jasmine Li STUDENTNUMBER}

\begin{document}
\maketitle

\section{Instruction and Summary}

\begin{enumerate}

    \item Which milestones were implemented? 

    Milestone 1, 2, 3
    
    % TODO: List the milestone(s) and in the case of 
    %       Milestones 4 & 5, list what features you 
    %       implemented, sorted into easy and hard 
    %       categories.

    \item How to view the game:
    % TODO: specify the pixes/unit, width and height of 
    %       your game, etc.  NOTE: list these details in
    %       the header of your breakout.asm file too!

    
    \begin{enumerate}

    \item Unit width in pixels:       8
\item Unit height in pixels:      8
\item Display width in pixels:    256
\item Display height in pixels:   256
\item Base Address for Display:   0x10008000


    \end{enumerate}

    

\begin{figure}[ht!]
    \centering
    % \includegraphics[width=0.3\textwidth]{name.png}
    \caption{caption}
    \label{Instructions}
\end{figure}

\begin{figure}[ht!]
    \centering
     \includegraphics[width=0.3\textwidth]{static_scene.png}
    \caption{Static Screen}
    \label{Static Screen}
\end{figure}


\item Game Summary:
% TODO: Tell us a little about your game.
\begin{itemize}
\item You can use the A, S, D keys to move the column left, down and right respectively. To shuffle the order of the gems, you can press W. The goal is to eliminate gems by arranging them so that there are 3 in a row of the same colour, either horizontally, vertically or diagonally. 
\item The features we implemented are: gravity
\end{itemize}

    
\end{enumerate}

\section{Attribution Table}
% TODO: If you worked in partners, tell us who was 
%       responsible for which features. Some reweighting 
%       might be possible in cases where one group member
%       deserves extra credit for the work they put in.

\begin{center}
\begin{tabular}{|| c | c ||}
\hline
 Student 1 (Cynthia 1011129832) &  Student 2 (Jasmine and student number) \\ 
 \hline
 Milestone 1, 2, 3 & Milestone 1, 2, 3\\
 \hline
 Task & Task\\
 \hline
 Task & Task\\ 
 \hline
 Task & Task\\ 
 \hline
 Task & Task\\
 \hline
 Task & Task\\  
 \hline
\end{tabular}
\end{center}

% TODO: Fill out the remainder of the document as you see 
%       fit, including as much detail as you think 
%       necessary to better understand your code. 
%       You can add extra sections and subsections to 
%       help us understand why you deserve marks for 
%       features that were more challenging than they
%       might initially seem.


\end{document}
